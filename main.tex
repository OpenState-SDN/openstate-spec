%!TEX TS-program = pdflatex

\documentclass[a4paper,12pt,oneside]{book}
\usepackage[utf8]{inputenc}
\usepackage[T1]{fontenc}
\usepackage[english]{babel}
\usepackage{mathtools}
\usepackage{graphicx}
\usepackage{listings}
\usepackage{fancyhdr} 
\usepackage[footnotesize,hang,bf]{caption} %img caption
\usepackage{tabularx}
\usepackage{setspace}
\usepackage{framed}
\usepackage{fancyvrb}
\usepackage{subfigure}
\usepackage{xcolor}
\usepackage{color}
\usepackage{lscape}
\usepackage{pdfpages}
\usepackage{acronym}
\usepackage{enumitem}
\usepackage{float}
\usepackage{amssymb}

%must be last package

\usepackage{hyperref}

%subitem
\newcommand{\revcomment}[1]{\textcolor{red}{ \bf [ #1 ]}}

%font per il linguaggio bash
\lstdefinestyle{BashInputStyle}{
  language=bash,
  basicstyle=\small\sffamily\scriptsize,
  frame=tb,
  columns=fullflexible,
  backgroundcolor=\color{yellow!20},
  linewidth=0.95\linewidth,
  xleftmargin=0.05\linewidth,
  breaklines=true
}

\lstdefinestyle{customc}{
  belowcaptionskip=1\baselineskip,
  breaklines=true,
  frame=L,
  xleftmargin=\parindent,
  language=C,
  showstringspaces=false,
  basicstyle=\footnotesize\ttfamily,
  keywordstyle=\bfseries\color{green!40!black},
  commentstyle=\itshape\color{purple!40!black},
  identifierstyle=\color{blue},
  stringstyle=\color{orange},
}


%Page layout
\topmargin 0cm 
\headsep 1cm 
\headheight 0.6cm
\textwidth 14.6cm
\textheight 21.8cm
\evensidemargin 1cm 
\oddsidemargin 1cm      

\fancyhead[L]{\chaptername \thechapter}
\fancyhead[LO]{\thesection} \fancyhead[RO]{\sectionmark}
\lhead{\slshape Ch. \thechapter}


% Configurazioni varie


% crea una nuova lunghezza e gli assegna un valore
\newlength{\defbaselineskip}
\setlength{\defbaselineskip}{\baselineskip}

% comando per settare  la variabile \baselineskip come multiplo di \defbaselineskip
\newcommand{\setlinespacing}[1]%
           {\setlength{\baselineskip}{#1 \defbaselineskip}}

% permette l'inserimento "rapido" del virgolettato (ad es. per citazioni)
\newcommand{\fastquote}[1]{``#1''}

% definizione dello stile delle didascalie delle immagini
\renewcommand{\captionfont}{%
	\normalfont \sffamily \slshape \footnotesize%
}


%%%%%%%%%%%%%%%%%%%%%%%%%%%%%%%%%%%%%%%%%%%%%%%%%%%%%%%%%%%%%%%%%%%%%
%                         Inizio Documento                          %
%%%%%%%%%%%%%%%%%%%%%%%%%%%%%%%%%%%%%%%%%%%%%%%%%%%%%%%%%%%%%%%%%%%%%

\hypersetup{pdfinfo={
	Title={Software-Defined Networking Applications Based on OpenState},
	Author={Luca Pollini,Davide Sanvito}
}}

\setcounter{secnumdepth}{3}
\setcounter{tocdepth}{3}
\begin{document}

%\include{front}
%\clearpage
%\newpage


% Settaggio interlinea
\setlinespacing{1.5}


% Inizio Numerazione Romana
\pagenumbering{Roman}

% Indice
\tableofcontents
\newpage
\clearpage

% Impostazioni della pagina
\pagestyle{fancy} 
\headsep=40pt 
\lhead{} 
\rhead{\slshape \leftmark} 
\cfoot{\thepage}


% Inizio Numerazione Araba
\pagenumbering{arabic}

% Vari Capitoli
% %!TEX root=main.tex

\chapter{Introduction}

\todo{Carmelo}{This section should give a clue of what OpenState is to someone who is at least confortable to the OpenFlow specification}

This document describes an extension to the OpenFlow specification v1.x () to enable support to stateful packet forwarding inside OpenFlow-enabled switches. Backward compatibility with OpenFlow is always guaranteed an exisitng elements and primitives are not modified in a way that breaks compatibility.

\begin{itemize}
	\item Motivation (Do we really need it?)
	\item Recall OpenFlow match/action flow table -> Stateless
	\item State machine abstraction
\end{itemize}


 %
\chapter{OpenState}
A packet entering an OpenFlow switch is processed through a set of linked flow tables that provide matching, forwarding, and packet modification.
We indicate with the term stateless stage the processing operated by a single flow table. Conversely, we define as stateful stage a logical block comprising a State Table and an XFSM table, and implementing our abstraction.
By default all the flow tables are intended as stateless. A new switch capability [\ref{sec:capability}] has been defined in order to support all the OpenState functionalities: flow states [\ref{chap:flow_states}] and global states [\ref{chap:global_states}]. The controller can enable stateful processing for one or more flow table by sending a special control message to the switch [\ref{sec:table_conf}] and by configuring the key extractors (lookup-scope and update-scope). A new state modify message called OFP\_STATE\_MOD has been defined to allow the controller to configure the state entries and key extractors [\ref{sec:msg_set_state}].
Finally two new actions OFPAT\_SET\_STATE [\ref{sec:act_set_state}] and OFPAT\_SET\_FLAGS [\ref{section:set_flag_action}] have been added in order to respectively implement and configure the XFSM state transitions in the flow table and configure the global states.
\chapter{Switch and table configuration}
\label{chap:configuration}

\section{Openstate capability}
\label{sec:capability}
A new OFPC\_OPENSTATE capability has been introduced.
The basic flow table data structure has been extended with a support data structure implementing the state table (a hash map indexed by the flow key), the lookup and update key extractor (two ordered lists of flow match TLV field indexes) and the global states.
By retrieving all the capabilities from the switch, the controller is able to properly configure the switch. If a switch is OpenState aware, OFPC\_OPENSTATE capability is defined enabling the controller to configure the statefulness of each stage by sending table feature message [\ref{sec:table_conf}].

\section{Stateful stage configuration}
\label{sec:table_conf}
If OFPTC\_TABLE\_STATEFUL flag is set, right after the packet headers are parsed, the flow state is retrieved and written in the state field, otherwise the packet directly jumps to the flow table. 


%!TEX root=000-main.tex

\chapter{Flow States}
\label{chap:flow_states}

\todo{??}{Refactoring. This section should contain an high level description of the necessary elements and primitives to handle flow states, no C struct nor protocol message specification should be given here. These should be included in the protocol section along with checks and errors.}

A flow-state is an information associated to a flow and shared among all the flow's packets. Flows are uniquelly defined by the lookup-scope and update-scope. TBC.

\section{State Table}

\todo{??}{

Describe the state table in terms of:
    \begin{itemize}
        \item columns (key, state label, timeouts, counters?)
        \item exact match
        \item table-miss (DEFAULT vs NULL)
        \item timeouts
    \end{itemize}
}

In case of table-miss (the key is not matched) then a \texttt{DEFAULT} state will be appended to the packet headers.

\subsection{Key Extractor}

\todo{??}{
Describe the key extractor in terms of:
    \begin{itemize}
        \item difference between lookup/update (we should discuss, see next comment)
        \item vector of header fields
        \item header-miss (what if the specified header doesn't exist)
    \end{itemize}
}

\comment{Carmelo}{It seems that the only case where I need to declare two distinct lookup-scope/update-scope is when I need to update the state for the reverse flow (e.g. MAC learning, reverse path forwarding consistency, etc.). Would it be better to define just one flow-scope (essentially the lookup scope) and give the possibility to call a set-state on a transformation of this flow scope? Example: I define the flow-scope as f=\{ipsrc,ipdst\} and I define a reverse() function, that returns the definition of the reverse flow for the passed scope. In this case it would be reverse(f) = \{ipdst,ipsrc\} }

If the header fields specified by the lookup-scope are not found (e.g. extracting the IP source address when the Ethernet type is not IP) or are only partially found, a special state value \texttt{NULL} is returned and no information about state is appended to the packet.

\section{Set-state Action}
\label{sec:act_set_state}

The \texttt{OFPAT\_SET\_STATE} action allows to set flow-states in a particulare stage of the pipeline.

The structure of the \texttt{OFPAT\_SET\_STATE} is defined in the following:

\begin{verbatim}
/* Action structure for OFPAT_SET_STATE. */
struct ofp_action_set_state {
    uint16_t type;     /* OFPAT_SET_STATE */
    uint16_t len;      /* Length is 16. */
    uint32_t state;    /* State value. */
    uint8_t stage_id;  /* Stage ID (same as flow table ID) of the state table */
    uint8_t pad[7];    /* Align to 64-bits. */
};
OFP_ASSERT(sizeof(struct ofp_action_set_state) == 16);
\end{verbatim}

\comment{Carmelo}{I think we should rename stage\_id into table\_id for consistency with the OpenFlow specification}
\comment{Luca, Davide}{\textbf{To Carmelo:} Since a stateful stage comprises a state table and and a flow table, we used stage\_id because we are not referring to the single flow table (table\_id) but we are referring to the entire stateful stage. }

\subsection{Atomicity}
As defined in OpenFlow, actions are usually executed at the end of the pipeline. The same applies for the \textbf{set-state} action, thus making the stateful steps ``lookup/update'' not atomic by default. Not enforcing atomicity can bring to consistency issues when more than one packet are processed by the pipeline at the same time. The only way to guarantees state consistency between packets is to call the \textbf{set-state} action from the \textbf{apply-action} instruction (instead of the \textbf{write-actions} instruction) in order to be sure to update the value contained in the state table when exiting a specific stage of the pipeline.

\subsection{Checks and Errors}

\begin{itemize}
\item Set-state action can be called only on stateful stage. This check is performed at action execution time (maybe the flow-mod message with a set-state action is received by the switch before configuring a stage as stateful. The important thing is that the stage is stateful at action execution time).
\comment{Luca, Davide}{Should we inform the controller about this error?}

\item Set-state action must be performed onto a stage with \texttt{stage\_id} less or equal than the number of pipeline’s tables. This check is performed at msg unpack time (the number of table is fixed, so installing a flow with a wrong action does not make sense). \todo{Luca, Davide}{}
\end{itemize}

\subsection{Priority}

The new \texttt{OFPAT\_SET\_STATE} action has to be executed with an higher priority with respect to the \texttt{OFPAT\_SET\_FIELD} action. Given an action set containing both a set field and a set state action, with this setting it is avoided that the set field modifies header fields used by the set state's update scope before the set state execution.

\comment{Carmelo}{What's the priority with regards to other actions such as output, drop, etc? I think we should explain this.}

\subsection{State Match Field}
\label{sec:match_state}

The \texttt{OXM\_OF\_STATE} field is the field used in the flow table to match on the state value defined in the virtual packet header field returned by a state table in a stateful stage. It is a 32 bit field.

\comment{Luca, Davide}{Does OXM\_OF\_STATE need to have a mask?}
\comment{Carmelo}{\textbf{To Luca, Davide:} I think yes, the state field should be maskable. An example use case could be matching on the second half of a MPLS label describing the ingress/egress switch. In the same way I propose to introduce the possibility to mask the state value in the set-state action.}

\begin{verbatim}
/* Flow state field definition (oxm-match.h) */
#define OXM_OF_STATE OXM_HEADER     (0x8000, 41, 4)
\end{verbatim}

\section{State Modification Messages}
\label{sec:msg_set_state}

OpenState defines four different types of \texttt{OFPT\_STATE\_MOD} messages: 
\begin{itemize}
\setlength\itemsep{0em}
\item \textbf{Set-lookup-extractor}: allows the controller to set the vector of header fields of the lookup-scope;
\item \textbf{Set-update-extractor}: allows the controller to set the vector of header fields of the update-scope;
\item \textbf{Set-flow-state command}: allows the controller to add/update a new row in the state table (equivalent to the set-state action);
\item \textbf{Delete-flow-state}: allows the controller to delete a row in the state table (equivalent to invoking a  set-flow-state command or a set-state action with \texttt{DEFAULT} state).
\end{itemize}

The structures used

\begin{verbatim}
/*
 * Controller to switch message  (add to enum ofp_type)
 */
OFPT_STATE_MOD = 30

/*
 * Max number of fields that can be used to compose
 * the key extractor vector.
 */
#define OFPSC_MAX_FIELD_COUNT 6

/*
 * Number of bytes composing the state key
 */
#define OFPSC_MAX_KEY_LEN 48


struct ofp_state_mod {
    struct   ofp_header header;
    uint64_t cookie;
    uint64_t cookie_mask;
    uint8_t  table_id;
    uint8_t  command;
    uint8_t  payload[];
};
\end{verbatim}

\comment{Luca, Davide}{
From OpenFlow spec:
\textit{The cookie field is an opaque data value chosen by the controller. This value appears in flow removed
messages and flow statistics, and can also be used to filter flow statistics, flow modification and flow deletion. It is not used by the packet processing pipeline. When a flow entry is inserted in a table through an OFPFC\_ADD message, its cookie field is set to the provided value. When a flow entry is modified (OFPFC\_MODIFY or OFPFC\_MODIFY\_STRICT messages), its cookie field is unchanged.}

Why has it been included in the ofp\_state\_mod message?
}

\begin{verbatim}
struct ofp_state_entry {
    uint32_t key_len;
    uint32_t state;
    uint8_t  key[OFPSC_MAX_KEY_LEN];
};

struct ofp_extraction {
    uint32_t field_count;
    uint32_t fields[OFPSC_MAX_FIELD_COUNT];
};

enum ofp_state_mod_command {
	OFPSC_SET_L_EXTRACTOR = 0,
	OFPSC_SET_U_EXTRACTOR,
	OFPSC_ADD_FLOW_STATE,	
	OFPSC_DEL_FLOW_STATE
 };

\end{verbatim}

\comment{Carmelo}{enum values are not defined for ofp\_state\_mod\_command}

\comment{Luca, Davide}{\textbf{To Carmelo:} An enumerator with no = defines its value by adding 1 to the value of the previous enumeration constant }

\comment{Luca, Davide}{who decides OFPSC\_MAX\_FIELD\_COUNT and OFPSC\_MAX\_KEY\_LEN? Should they be configurable?}

\subsubsection{Set-lookup-extractor command}
\label{subsec:set_l_extr}

The \texttt{OFPSC\_SET\_L\_EXTRACTOR} command gives the opportunity to set the lookup scope used in the state table.
The controller sends a \texttt{OFPT\_STATE\_MOD} message with the following parameters:

\begin{itemize}
\item table\_id = ID of the stage to be setup
\item command = 0 (\texttt{OFPSC\_SET\_L\_EXTRACTOR})
\item field\_count = number of specified field 
\item fields = key’s fields
\end{itemize}

\todo{Luca, Davide}{Include C struct of OFPSC\_SET\_L\_EXTRACTOR}
\comment{Luca, Davide}{A Set-lookup-extractor command is a ofp\_state\_mod message with command = 0 and struct ofp\_extraction as ofp\_state\_mod's payload}

\subsubsection{Set-update-extractor command}
\label{subsec:set_u_extr}

The OFPSC\_SET\_U\_EXTRACTOR command gives the opportunity to set the update scope used in the state table.
The controller sends a \texttt{OFPT\_STATE\_MOD} message with the following parameters:

\begin{itemize}
\item table\_id = ID of the stage to be setup
\item command = 1 (OFPSC\_SET\_U\_EXTRACTOR)
\item field\_count = number of specified field 
\item fields = key’s fields
\end{itemize}

\todo{Luca, Davide}{Include C struct of OFPSC\_SET\_U\_EXTRACTOR}
\comment{Luca, Davide}{A Set-update-extractor command is a ofp\_state\_mod message with command = 1 and struct ofp\_extraction as ofp\_state\_mod's payload}

\subsection{Set-flow-state command}
\label{subsec:add_flow}

The OFPSC\_ADD\_FLOW\_STATE command gives the opportunity to insert (or to overwrite) an entry in the state table.
The controller sends a \texttt{OFPT\_STATE\_MOD} message with the following parameters:

\begin{itemize}
\item table\_id = ID of the stage to be modified
\item command = 2 (OFPSC\_ADD\_FLOW\_STATE)
\item key\_count = it is the key size in byte
\item state = it is the state to insert in the state table
\item keys = key splitted in bytes (e.g: ip 10.0.0.1 is stored as [10,0,0,1])
\end{itemize}

\todo{Luca, Davide}{Include C struct of OFPSC\_ADD\_FLOW\_STATE}
\comment{Luca, Davide}{A Set-flow-state command is an ofp\_state\_mod message with command = 2 and struct ofp\_state\_entry as ofp\_state\_mod's payload}

\comment{Carmelo}{OFPSC\_ADD\_FLOW\_STATE should be renamed in OFPSC\_SET\_FLOW\_STATE}

\subsection{Delete-flow-state command}
\label{subsec:del_flow}

With the OFPSC\_DEL\_FLOW\_STATE command is possible to delete a state table's entry.
The controller sends a \texttt{OFPT\_STATE\_MOD} message with the following parameters:

\begin{itemize}
\item table\_id = ID of the stage to be modified
\item command = 3 (OFPSC\_DEL\_FLOW\_STATE)
\item key\_count = it is the key size in byte
\item state = ANY
\item keys = key splitted in bytes (e.g: ip 10.0.0.1 is stored as [10,0,0,1])
\end{itemize}

\todo{Luca, Davide}{Include C struct of OFPSC\_DEL\_FLOW\_STATE}
\comment{Luca, Davide}{A Delete-flow-state command is an ofp\_state\_mod message with command = 3 and struct ofp\_state\_entry as ofp\_state\_mod's payload}

The state value is not taken in consideration because the state table simply uses as key the lookup-scope's fields to delete the entry with key keys.

\paragraph{Checks and Errors}

\begin{itemize}
\item In Set-lookup-extractor/Set-update-extractor commands \texttt{field\_count} must be consistent with the number of fields provided in \texttt{fields[]}, otherwise an error (\texttt{OFPET\_BAD\_ACTION}, \texttt{OFPBAC\_BAD\_LEN}) is returned at message unpack time
\comment{Carmelo}{Are both \texttt{OFPET\_BAD\_ACTION} and \texttt{OFPBAC\_BAD\_LEN} returned?}
\comment{Luca, Davide}{\textbf{To Carmelo:} An error message has a type (\texttt{OFPET\_BAD\_ACTION}) and a code (\texttt{OFPBAC\_BAD\_LEN}): }

\item In Set-flow-state/Delete-flow-state commands \texttt{key\_count} must be consistent with the number of keys provided in \texttt{key[]}, otherwise an error (\texttt{OFPET\_BAD\_ACTION}, \texttt{OFPBAC\_BAD\_LEN}) is returned at msg unpack time. 

\item In Set-flow-state/Delete-flow-state commands \texttt{key\_count} must be consistent with the number of fields of the update-scope (defined before with a OFPSC\_SET\_U\_EXTRACTOR). This check is performed at msg execution time
\comment{Luca, Davide}{Should we inform the controller about this error?}

\item Lookup-scope and update-scope should provide same length keys. This check should be performed by the switch when a set-extractor message is received: if we have already set the other extractor, the new extractor must have the same length. The check is performed at msg execution time.
\comment{Luca, Davide}{Should we inform the controller about this error?}
\todo{Luca, Davide}

\item Set-flow-state/Delete-flow-state commands must be executed onto a stage with stage\_id less or equal than the number of pipeline’s tables, otherwise the switch returns an error (OFPET\_BAD\_REQUEST, OFPBRC\_BAD\_TABLE\_ID). This check is performed at msg unpack time (since the number of table is fixed).

\end{itemize}
%!TEX root=main.tex

\chapter{Global States}
\label{chap:global_states}
By extending the flow state concept some states could be shared among multiple flows. For this reason global states have been developed. These states (a.k.a. flags) are defined at datapath level and are not related to a single flow of a particular stage. Now each incoming flow’s packet can be matched also according to the current value of global states. Global states can be updated c by means of a new action ``set-flags'' triggered by a match in the flow table. Furthermore, the controller is able to modify and reset global states value of a specific switch exploiting the new flag modification messages.

\todo{Luca, Davide}{Decription of the global states as a string of bits, maskable. Who defines the maximum length of the string? the switch? The specification?}

\section{Flag Modification Messages}
\label{sec:flag_mod_msg}

The following types of flag modification messages are defined:

\begin{itemize}
	\item \textbf{Set-flags}: allows the controller to update the value of global states. Values can be totally overwritten or, by using a mask, selectively modified.
	\item \textbf{Reset-flags}: allows the controller to reset global states to the default value.
	\comment{Carmelo}{Who defines the default value? The switch? The controller?}
\end{itemize}

\section{Set-flag Action}
\label{sec:act_set_flag}

In addition to flag modification messages \ref{sec:flag_mod_msg}, the global states can be modified as a consequence of packet matching in a flow entry. By adding a set-flag action to the action set, it is possible to modify the global state values in any stage of the pipeline. OpenFlow allows to execute actions in the group table, so it is possible to update global state values from the group table by inserting a set-flag action in the action bucket. Using the set-flag action values can be totally overwritten or, by using a mask, selectively modified.

% Fine del documento

\end{document}

