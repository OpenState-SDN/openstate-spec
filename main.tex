%!TEX TS-program = pdflatex

\documentclass[a4paper,12pt,oneside]{book}
\usepackage[utf8]{inputenc}
\usepackage[T1]{fontenc}
\usepackage[english]{babel}
\usepackage{mathtools}
\usepackage{graphicx}
\usepackage{listings}
\usepackage{fancyhdr} 
\usepackage[footnotesize,hang,bf]{caption} %img caption
\usepackage{tabularx}
\usepackage{setspace}
\usepackage{framed}
\usepackage{fancyvrb}
\usepackage{subfigure}
\usepackage{xcolor}
\usepackage{color}
\usepackage{lscape}
\usepackage{pdfpages}
\usepackage{acronym}
\usepackage{enumitem}
\usepackage{float}
\usepackage{amssymb}

%must be last package

\usepackage{hyperref}

%subitem
\newcommand{\revcomment}[1]{\textcolor{red}{ \bf [ #1 ]}}

%font per il linguaggio bash
\lstdefinestyle{BashInputStyle}{
  language=bash,
  basicstyle=\small\sffamily\scriptsize,
  frame=tb,
  columns=fullflexible,
  backgroundcolor=\color{yellow!20},
  linewidth=0.95\linewidth,
  xleftmargin=0.05\linewidth,
  breaklines=true
}

\lstdefinestyle{customc}{
  belowcaptionskip=1\baselineskip,
  breaklines=true,
  frame=L,
  xleftmargin=\parindent,
  language=C,
  showstringspaces=false,
  basicstyle=\footnotesize\ttfamily,
  keywordstyle=\bfseries\color{green!40!black},
  commentstyle=\itshape\color{purple!40!black},
  identifierstyle=\color{blue},
  stringstyle=\color{orange},
}


%Page layout
\topmargin 0cm 
\headsep 1cm 
\headheight 0.6cm
\textwidth 14.6cm
\textheight 21.8cm
\evensidemargin 1cm 
\oddsidemargin 1cm      

\fancyhead[L]{\chaptername \thechapter}
\fancyhead[LO]{\thesection} \fancyhead[RO]{\sectionmark}
\lhead{\slshape Ch. \thechapter}


% Configurazioni varie


% crea una nuova lunghezza e gli assegna un valore
\newlength{\defbaselineskip}
\setlength{\defbaselineskip}{\baselineskip}

% comando per settare  la variabile \baselineskip come multiplo di \defbaselineskip
\newcommand{\setlinespacing}[1]%
           {\setlength{\baselineskip}{#1 \defbaselineskip}}

% permette l'inserimento "rapido" del virgolettato (ad es. per citazioni)
\newcommand{\fastquote}[1]{``#1''}

% definizione dello stile delle didascalie delle immagini
\renewcommand{\captionfont}{%
	\normalfont \sffamily \slshape \footnotesize%
}


%%%%%%%%%%%%%%%%%%%%%%%%%%%%%%%%%%%%%%%%%%%%%%%%%%%%%%%%%%%%%%%%%%%%%
%                         Inizio Documento                          %
%%%%%%%%%%%%%%%%%%%%%%%%%%%%%%%%%%%%%%%%%%%%%%%%%%%%%%%%%%%%%%%%%%%%%

\hypersetup{pdfinfo={
	Title={OpenState specification},
	Author={Carmelo Cascone,Luca Pollini,Davide Sanvito}
}}

\setcounter{secnumdepth}{3}
\setcounter{tocdepth}{3}
\begin{document}

%\include{front}
%\clearpage
%\newpage


% Settaggio interlinea
\setlinespacing{1.5}


% Inizio Numerazione Romana
\pagenumbering{Roman}

% Indice
\tableofcontents
\newpage
\clearpage

% Impostazioni della pagina
\pagestyle{fancy} 
\headsep=40pt 
\lhead{} 
\rhead{\slshape \leftmark} 
\cfoot{\thepage}


% Inizio Numerazione Araba
\pagenumbering{arabic}

% Vari Capitoli
% %!TEX root=000-main.tex

\chapter{Introduction}
This document is intended to be an extension to the OpenFlow specification v1.x (???). Backward compatibility is guaranteed.

\todo{Carmelo}{This section should give a clue of what OpenState is to someone who is at least confortable to the OpenFlow specification}

 %
\chapter{OpenState}
A packet entering an OpenFlow switch is processed through a set of linked flow tables that provide matching, forwarding, and packet modification.
We indicate with the term stateless stage the processing operated by a single flow table. Conversely, we define as stateful stage a logical block comprising a State Table and an XFSM table, and implementing our abstraction.
By default all the flow tables are intended as stateless. A new switch capability [\ref{sec:capability}] has been defined in order to support all the OpenState functionalities: flow states [\ref{chap:flow_states}] and global states [\ref{chap:global_states}]. The controller can enable stateful processing for one or more flow table by sending a special control message to the switch [\ref{sec:table_conf}] and by configuring the key extractors (lookup-scope and update-scope). A new state modify message called OFP\_STATE\_MOD has been defined to allow the controller to configure the state entries and key extractors [\ref{sec:msg_set_state}].
Finally two new actions OFPAT\_SET\_STATE [\ref{sec:act_set_state}] and OFPAT\_SET\_FLAGS [\ref{section:set_flag_action}] have been added in order to respectively implement and configure the XFSM state transitions in the flow table and configure the global states.
\include{configuration}
%!TEX root=main.tex

\chapter{Flow States}
\label{chap:flow_states}

\section{Flow Identification}

Flow states are associated with packets and are valid only inside that stateful stage that produced them. Inside a stateful stage, flows can be arbitrary defined by using ``flow scopes'', which can be seen as the vector of header fields that distinguish one flow from another. For example a Layer 2 flow can be defined by using just the MAC source address and MAC destination address (2 fields), while a flow in the socket sense can be defined by using the whole L2-L4 header (6 fields).

In OpenState states for a given flow can be updated by events occurring on {\em different} flows. A prominent example is MAC learning: packets are forwarded using the {\em destination} MAC address, but the forwarding database is updated using the {\em source} MAC address. Similarly, the handling of bidirectional flows may encounter the same needs; for instance, the detection of a returning TCP SYNACK packet could trigger a state transition on the opposite direction. And in protocols such as FTP, a control exchange on port 21 could be used to set a state on the data transfer session on port 20. For this reason, two types of flow scopes are defined, the ``lookup-scope'' and the ``update-scope'', as the ordered sequence of header fields that shall be used to produce the key used to access the state table and perform, respectively, a lookup or an update operation.

\comment{Carmelo}{We should allow masked header fields in the flow scope. Example is if we wanna use only the IP subnet to retrieve the state or just part of the MPLS label (for example the first 10 bits that describes the ingress switch)}

The lookup-scope and the update-scope are intrinsic to the state table and are used to configure the key extraction process.

\comment{Carmelo}{It seems that the only case where I need to declare two distinct lookup-scope/update-scope is when I need to update the state for the reverse flow (e.g. MAC learning, reverse path forwarding consistency, etc.). Would it be better to define just one flow-scope (essentially the lookup scope) and give the possibility to call a set-state on a transformation of this flow scope? Example: I define the flow-scope as f=\{ipsrc,ipdst\} and I define a reverse() function, that returns the definition of the reverse flow for the passed scope. In this case it would be reverse(f) = \{ipdst,ipsrc\} }

\section{State Table}

\begin{table}[h]
    \centering
    \begin{tabular}{| c | c | c | c |}
        \hline
        Key & State & Timeouts \\
        \hline
    \end{tabular}
    \caption{Main components of a state entry in the state table}
    \label{t:state-entry}
\end{table}

A state table consists of state entries. Each state table entry (see Fig.\ref{t:state-entry}) contains:

\begin{itemize}
    \item \textbf{Key:} String of bit used to match the packet flow key obtained from the key extractor;
    \item \textbf{State:} value associated with a specific flow key
    \item\textbf{Timeouts:} Maximum amount of time or idle time before the entry is expired by the switch;
\end{itemize}

The match on the state table is performed using the key extracted using the lookup-scope, and it is performed exactly, in other words wildcards are not allowed. In case of a table-miss (the key is not found) then a \emph{DEFAULT} state is appended to the packet headers. If the header fields specified by the lookup-scope are not found (e.g. extracting the IP source address when the Ethernet type is not IP), a special state value \emph{NULL} is returned.

\comment{Carmelo}{What is the maximum length in bits of the state? Should it be defined by the switch? Should it be programmable?}

\subsection{Timeouts}

\todo{}{Describe timeouts and implementation}

If the header fields specified by the update-scope are not found in the packet, the set-state action is not executed.

\subsection{State Modification Messages}

\label{sec:msg_set_state}

4 different state modification messages are defined by OpenState:

\begin{itemize}
    \item \textbf{Set-lookup-extractor:} allows the controller to set the header fields vector for the lookup-scope of the state table.
    \item \textbf{Set-update-extractor:} allows the controller to set the header fields vector for the update-scope of the state table.
    \item \textbf{Set-flow-state:} allows the controller to add or update a state entry in the state table.
    \item \textbf{Delete-flow-state:} allows the controller to delete a state entry in the state table. This command is equivalent to invoking a set-flow-state command or a set state action \ref{sec:act_set_state} with \texttt{DEFAULT} state.
\end{itemize}

\comment{Luca, Davide}{There is a little difference between the delete command and a set state action with default state. The delete command removes the entry from the state table saving space, while by using a set state action with a DEFAULT state, the state entry is not removed. The two commands can be considered equivalent thanks to the fact that a default state will always be associated to an unknown flow.}
\comment{Carmelo}{We might consider to delete an entry when we set its state do DEFAULT.}

\section{Set-state Action}
\label{sec:act_set_state}
In addition to state modification messages \ref{sec:msg_set_state}, state transitions can be triggered as a consequence of packet matching in a flow entry. By adding a set-state action to the action set, it is possible to execute state transitions in the same stage or in any other (stateful) stage of the pipeline. Multiple state transitions are allowed by defining more than one set-state action in the action set. OpenFlow allows to execute actions in the group table, so it is possible to perform state transitions from the group table by inserting a set-state action in the action bucket.

When the switch executes a set-state action, the packet header is processed by the update-scope key extractor of the specific state table, the corresponding entry is then updated.

\subsection{Atomicity}
As defined in OpenFlow, actions are usually executed at the end of the pipeline. The same applies for the set-state action, thus making the stateful steps ``lookup/update'' not atomic by default. Not enforcing atomicity can bring to consistency issues when more than one packet are processed by the pipeline at the same time. The only way to guarantees state consistency between packets is to call the set-state action from the apply-action instruction (instead of the write-actions instruction) in order to be sure to update the value contained in the state table when exiting a specific stage of the pipeline.
%!TEX root=000-main.tex

\chapter{Global States}
\label{chap:global_states}
Global states (or flags) are defined at datapath level and are not related to a single flow of a particular stage.
Flags are 32 boolean registers and are grouped in an uint32\_t variable global\_states in the datapath struct.

\section{Flags match field}
\label{section:oxm_of_flags}

If a switch supports OpenState (flag OFPTC\_TABLE\_STATEFUL set), right after the packet headers are parsed, the global states are retrieved and written in the flags field. OXM\_OF\_FLAGS is a field with mask, so it is possible to match it either exactly or with wildcards. A 0 bit in the mask means i-th flags value is ``do not care'', while a 1 bit value means ``exact match''.

Definition in ofsoftswitch13 (oxm-match.h file)
\begin{verbatim}
/* Global States */
#define OXM_OF_FLAGS OXM_HEADER     (0x8000, 40, 4)
#define OXM_OF_FLAGS_W OXM_HEADER_W (0x8000, 40, 4)
\end{verbatim}
Example match:
\begin{verbatim}
flags=(4,5)
\end{verbatim}
This command allows to match over *****************************1*0 flags configuration (4 in binary is 100 and the mask 5 is 101 that is exact match on LSB 1 (0 value) and LSB 3 (1 value) and ``don’t care'' over all the other flags. In order to perform an exact match on flags value no mask is required.
\\Example match:
\begin{verbatim}
flags=4
\end{verbatim}
NB: this match is very differend from the previous one. With this command we are matching over 00000000000000000000000000000100 flags configuration, so it is an exact match.

\section{Set flag action}
\label{section:set_flag_action}
The OFPAT\_SET\_FLAG action is used to set flags' value.
In the flow modification message, the controller defines the action with the following parameters:
\begin{itemize}
\setlength\itemsep{0em}
\item flag = flags value
\item mask = flags mask
\end{itemize}

\paragraph{Structures in ofsoftswitch13:}\mbox{}\\
\begin{verbatim}
OFPAT_SET_FLAG = 29,   /* Set a single flag value of the global state */

/* Action structure for OFPAT_SET_FLAG */
struct ofp_action_set_flag {
    uint16_t type; /* OFPAT_SET_FLAG */
    uint16_t len;  /* Length is 8. */
    uint32_t value; /* flag value */
    uint32_t mask;    /*flag mask*/
    uint8_t pad[4];   /* Align to 64-bits. */
};
OFP_ASSERT(sizeof(struct ofp_action_set_flag) == 16);
\end{verbatim}

\section{Flag modification messages}
\label{section:flag_mod_msg}
The OFPT\_FLAG\_MOD message allow the controller to modify global states value.
\paragraph{Structures in ofsoftswitch13:}\mbox{}\\
\begin{verbatim}
OFPT_FLAG_MOD = 31,  /* Controller/switch message */

struct ofp_flag_mod {
    struct ofp_header header;
    uint32_t flag;
    uint32_t flag_mask;
    uint8_t command;
    uint8_t pad[7];                  /* Pad to 64 bits. */
};

enum ofp_flag_mod_command { 
    OFPSC_MODIFY_FLAGS = 0,
    OFPSC_RESET_FLAGS
};

\end{verbatim}

There are two message types.
\subsection{Modify flags command}
The OFPSC\_MODIFY\_FLAGS allows to modify global states values. The controller sends a OFPSC\_MODIFY\_FLAGS message with the following parameters:
\begin{itemize}
\setlength\itemsep{0em}
\item flag = flags value
\item flag\_mask = mask value
\item command = 0 (OFPSC\_MODIFY\_FLAGS)
\end{itemize}

\subsection{Reset flags command}
The OFPSC\_MODIFY\_FLAGS allows to reset global states to the default value (OFP\_GLOBAL\_STATES\_DEFAULT in openflow.h)
The controller sends a OFPSC\_MODIFY\_FLAGS message with the following parameters:
\begin{itemize}
\setlength\itemsep{0em}
\item flag = ANY
\item flag\_mask = ANY
\item command = 1 (OFPSC\_RESET\_FLAGS)
\end{itemize}

% Fine del documento

\end{document}

