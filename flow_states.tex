%!TEX root=000-main.tex

\chapter{Flow States}
\label{chap:flow_states}

\todo{??}{Refactoring. This section should contain an high level description of the necessary elements and primitives to handle flow states, no C struct nor protocol message specification should be given here. These should be included in the protocol section along with checks and errors.}

A flow-state is an information associated to a flow and shared among all the flow's packets. Flows are uniquelly defined by the lookup-scope and update-scope. TBC.

\section{State Table}

\todo{??}{

Describe the state table in terms of:
    \begin{itemize}
        \item columns (key, state label, timeouts, counters?)
        \item exact match
        \item table-miss (DEFAULT vs NULL)
        \item timeouts
    \end{itemize}
}

In case of table-miss (the key is not matched) then a \texttt{DEFAULT} state will be appended to the packet headers.

\subsection{Key Extractor}

\todo{??}{
Describe the key extractor in terms of:
    \begin{itemize}
        \item difference between lookup/update (we should discuss, see next comment)
        \item vector of header fields
        \item header-miss (what if the specified header doesn't exist)
    \end{itemize}
}

\comment{Carmelo}{It seems that the only case where I need to declare two distinct lookup-scope/update-scope is when I need to update the state for the reverse flow (e.g. MAC learning, reverse path forwarding consistency, etc.). Would it be better to define just one flow-scope (essentially the lookup scope) and give the possibility to call a set-state on a transformation of this flow scope? Example: I define the flow-scope as f=\{ipsrc,ipdst\} and I define a reverse() function, that returns the definition of the reverse flow for the passed scope. In this case it would be reverse(f) = \{ipdst,ipsrc\} }

If the header fields specified by the lookup-scope are not found (e.g. extracting the IP source address when the Ethernet type is not IP) or are only partially found, a special state value \texttt{NULL} is returned and no information about state is appended to the packet.


