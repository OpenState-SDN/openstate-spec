%!TEX root=000-main.tex

\chapter{Flow States}
\label{chap:flow_states}

\todo{??}{Refactoring. This section should contain an high level description of the necessary elements and primitives to handle flow states, no C struct nor protocol message specification should be given here. These should be included in the protocol section along with checks and errors.}

A flow-state is an information associated to a flow and shared among all the flow's packets. Flows are uniquelly defined by the lookup-scope and update-scope. TBC.

\section{State Table}

\todo{??}{

Describe the state table in terms of:
    \begin{itemize}
        \item columns (key, state label, timeouts, counters?)
        \item exact match
        \item table-miss (DEFAULT vs NULL)
        \item timeouts
    \end{itemize}
}

In case of table-miss (the key is not matched) then a \texttt{DEFAULT} state will be appended to the packet headers.

\subsection{Key Extractor}

\todo{??}{
Describe the key extractor in terms of:
    \begin{itemize}
        \item difference between lookup/update (we should discuss, see next comment)
        \item vector of header fields
        \item header-miss (what if the specified header doesn't exist)
    \end{itemize}
}

\comment{Carmelo}{It seems that the only case where I need to declare two distinct lookup-scope/update-scope is when I need to update the state for the reverse flow (e.g. MAC learning, reverse path forwarding consistency, etc.). Would it be better to define just one flow-scope (essentially the lookup scope) and give the possibility to call a set-state on a transformation of this flow scope? Example: I define the flow-scope as f=\{ipsrc,ipdst\} and I define a reverse() function, that returns the definition of the reverse flow for the passed scope. In this case it would be reverse(f) = \{ipdst,ipsrc\} }

If the header fields specified by the lookup-scope are not found (e.g. extracting the IP source address when the Ethernet type is not IP) or are only partially found, a special state value \texttt{NULL} is returned and no information about state is appended to the packet.
If the header fields specified by the update-scope are not found in the packet, the set-state action is not executed.

\subsection{State Modification Messages}

\label{sec:msg_set_state}

State modification messages can have the following types:
\scriptsize\begin{verbatim}
enum ofp_state_mod_command {
    OFPSC_SET_L_EXTRACTOR = 0,
    OFPSC_SET_U_EXTRACTOR,
    OFPSC_ADD_FLOW_STATE,   
    OFPSC_DEL_FLOW_STATE
 };
\end{verbatim}\normalsize


\comment{Carmelo}{enum values are not defined for ofp\_state\_mod\_command}
\comment{Luca, Davide}{\textbf{To Carmelo:} An enumerator with no = defines its value by adding 1 to the value of the previous enumeration constant }

\noindent
To \textbf{setup the lookup extractor} the controller has to set the \texttt{OFPSC\_SET\_L\_EXTRACTOR} command in a state modification message. 
This command allows the controller to set the vector of lookup-scope's header fields used in the state table.
\\\\The controller can \textbf{setup the update extractor} by sending a state modification message with the \texttt{OFPSC\_SET\_U\_EXTRACTOR} command set.
This command gives the opportunity to set the vector of update-scope's header fields used in the state table.
\\\\To \textbf{add} or \textbf{update} a state entry in the state table, the controller has to send a state modification message with the \texttt{OFPSC\_ADD\_FLOW\_STATE} command set. This message is equivalent to the set state action \ref{sec:act_set_state}.
\\\\The controller can \textbf{delete} a state entry by means of a state modification message with the \texttt{OFPSC\_DEL\_FLOW\_STATE} command set. This command is equivalent to invoking a \texttt{OFPSC\_ADD\_FLOW\_STATE} command or a set state action \ref{sec:act_set_state} with \texttt{DEFAULT} state.

\comment{Luca, Davide}{There is a little difference between the delete command and a set state action with default state. The delete command removes the entry from the state table saving space, while by using a set state action with a DEFAULT state, the state entry is not removed. The two commands can be considered equivalent thanks to the fact that a default state will always be associated to an unknown flow.}

\section{Set-state Action}
\label{sec:act_set_state}
In addition to state modification messages \ref{sec:msg_set_state}, state transitions can be triggered as a consequence of packet matching in a flow entry. By adding a set-state action to the action set, it is possibile to execute state transitions in the same stage or in any other (stateful) stage of the pipeline. Multiple state transitions are allowed by defining more than one set-state action in the action set.
OpenFlow allows to execute actions in the group table, so it is possible to perform state transitions from the group table by inserting a set-state action in the action bucket.

\subsection{Atomicity}
As defined in OpenFlow, actions are usually executed at the end of the pipeline. The same applies for the \textbf{set-state} action, thus making the stateful steps ``lookup/update'' not atomic by default. Not enforcing atomicity can bring to consistency issues when more than one packet are processed by the pipeline at the same time. The only way to guarantees state consistency between packets is to call the \textbf{set-state} action from the \textbf{apply-action} instruction (instead of the \textbf{write-actions} instruction) in order to be sure to update the value contained in the state table when exiting a specific stage of the pipeline.